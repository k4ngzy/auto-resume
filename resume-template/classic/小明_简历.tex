% !TEX TS-program = xelatex
% !TEX encoding = UTF-8 Unicode
% !Mode:: "TeX:UTF-8"

\documentclass{resume}
\usepackage{zh_CN-Adobefonts_external}
\usepackage{linespacing_fix}
\usepackage{cite}
\usepackage{graphicx}
\usepackage{tabularray}

\begin{document}
\pagenumbering{gobble}

\name{小明}
\contactInfo{性别}{年龄}{籍贯}{}
\contactInfo{电话}{意向岗位}{邮箱}{}

\section{教育背景}
\datedsubsection{\textbf{xx大学},计算机科学与技术,\textit{硕士}}{2024.09 - 2027.06}
\datedsubsection{\textbf{xx大学},信息与计算科学,\textit{学士}}{2020.09 - 2024.06}

\section{技术特长}
\begin{itemize}[parsep=0.2ex]
  \item PyTorch、Python、CUDA
  \item 大模型训练:熟悉 DeepSpeed 分布式训练框架,掌握 LoRA 微调方法,具备混合精度训练及显存优化实践经验
  \item 推理优化:使用 vLLM 和 Triton 构建高性能推理服务,开展过并发测试与性能调优
  \item NLP 技术栈:深入理解 Transformer 架构,实践过 Prompt Engineering 与 RAG 相关技术,熟悉对话系统构建与生成模型评估方法
\end{itemize}

\section{项目经历}
\datedsubsection{\textbf{大规模中文情感分析系统},角色: 负责人}{2024.10 - 2025.03}
\begin{itemize}
  \item 负责搭建电商和社交媒体场景的中文情感分析系统。
  \item 基于 RoBERTa-wwm 训练情感分类模型,并加入领域词表提升效果。
  \item 尝试使用 Qwen 进行 prompt 和 LoRA 微调,以改进长文本与隐含情绪的识别能力。
  \item 将大模型能力蒸馏到小模型以提升推理速度,并将系统部署在 Triton 与 vLLM 上完成基础测试。
  \item 最终模型效果与速度较原方案均有提升。
\end{itemize}
\datedsubsection{\textbf{智能客服意图识别与多轮对话系统},角色: 核心成员}{2025.04 - 2025.09}
\begin{itemize}
  \item 参与线上教育平台的智能客服系统开发,目标是实现多轮对话、意图识别和任务执行。
  \item 参与意图识别和知识问答部分的设计,尝试加入 RAG 来提升问答表现。
  \item 使用 LLaMA 的微调方法改进指令执行效果,使用工具优化显存和训练速度。
  \item 系统加入对话记忆和拒答策略,整体多轮对话的成功率和意图识别表现明显提升,并在实际平台稳定运行。
\end{itemize}

\section{荣誉证书}
\begin{itemize}[parsep=0.2ex]
  \item 2024 年获全国人工智能创新竞赛一等奖
  \item 2025 年获研究生国家奖学金
\end{itemize}

\end{document}
